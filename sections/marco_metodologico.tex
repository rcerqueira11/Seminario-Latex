\chapter{Marco Metodológico}


\section{Metodología Tradicional} 
\subsection{RUP}
<<<<<<< HEAD

 RUP es un proceso para el desarrollo de un proyecto de software que defineclaramente quien, cómo, cuándo y qué debe hacerse en el proyecto, con 3características esenciales, está dirigido por:

\begin{itemize}

    \item Los Casos de Uso : que orientan elproyecto a la importancia para el usuario y lo que este quiere. 

    \item La arquitectura: que Relaciona la toma de decisiones que indican cómo tiene queser construido el sistema y en qué orden.

    \item Iterativo e incremental: dividiéndose el proyecto en mini proyectos donde los casos de uso y laarquitectura cumplen sus objetivos de manera más depurada.

\end{itemize}


\subsubsection{Ciclo de vida}
RUP divide el proceso en 4 fases, dentro de las cuales se realizan variasiteraciones en número variable según el proyecto y en las que se hace un mayor omenor hincapié en los distintas actividades
=======
\subsubsection{Ciclo de vida}

>>>>>>> 488d646cf069e89dc1d33fc9ebdc703497d7b8e6
\begin{figure}[!htb]
	\minipage{0.15\textwidth}
	\includegraphics[width=\linewidth]{img/fases-rup.jpg}
	\endminipage\hfill
	
\end{figure}


\begin{itemize}

    \item Fase de Inicio: Esta fase tiene como propósito definir y acordar el alcance del proyecto con los patrocinadores, identificar los riesgos asociados al proyecto, proponer una visión muy general de la arquitectura de software y producir el plan de las fases y el de iteraciones posteriores.

	\item Fase de elaboración: En la fase de elaboración se seleccionan los casos de uso que permiten definir la arquitectura base del sistema y se desarrollaran en esta fase, se realiza la especificación de los casos de uso seleccionados y el primer análisis del dominio del problema, se diseña la solución preliminar.

	\item Fase de Desarrollo: El propósito de esta fase es completar la funcionalidad del sistema, para ello se deben clarificar los requisitos pendientes, administrar los cambios de acuerdo a las evaluaciones realizados por los usuarios y se realizan las mejoras para el proyecto.

	\item Fase de Transición: El propósito de esta fase es asegurar que el software esté disponible para los usuarios finales, ajustar los errores y defectos encontrados en las pruebas de aceptación, capacitar a los usuarios y proveer el soporte técnico necesario. Se debe verificar que el producto cumpla con las especificaciones entregadas por las personas involucradas en el proyecto.	

\end{itemize}
\subsection{Cuadro Comparativo}

\section{Metodología Ágil} 

El desarrollo ágil de software envuelve un enfoque para la toma de decisiones en los proyectos basados en el desarrollo iterativo e incremental, donde los requisitos y soluciones evolucionan con el tiempo según la necesidad del proyecto. Así el trabajo es realizado mediante la colaboración de equipos auto-organizados y multidisciplinarios, inmersos en un proceso compartido de toma de decisiones a corto plazo.

\subsection{SCRUM}
\subsubsection{Ciclo de vida}
<<<<<<< HEAD


\subsubsection{Spring}
\setlength{\parskip}{5mm}

	Es el periodo de tiempo durante el que se desarrolla un incremento de funcionalidad dura maximo 30 dias. Constituye el núcleo de Scrum, que divide de esta forma el desarrollo de un proyecto en un conjunto de pequeñas “carreras”. Durante el proceso no se puede modificar el trabajo que se ha acordado en el Backlog, solo es posible cambiar el curso de un sprint abortandolo y esto solo puede hacerlo el Scrum Master. 
	
\setlength{\parskip}{0mm}

\subsubsection{}
    

=======
>>>>>>> 488d646cf069e89dc1d33fc9ebdc703497d7b8e6
\subsection{XP}
\subsubsection{Ciclo de vida}
\subsection{}
\subsection{Cuadro Comparativo}

\section{Cuadro comparativo [Metodologías Tradicionales vs Ágiles]} 
<<<<<<< HEAD

=======
>>>>>>> 488d646cf069e89dc1d33fc9ebdc703497d7b8e6
